 \chapter{Conclusiones}
\label{chap:conclusiones}

En esta tesis se analiz\'o la problem\'atica de los desechos espaciales, con foco central en los riesgos de colisi\'on con misiones operativas. El estudio de un acercamiento de riesgo o encuentro, se basa en los resultados de las propagaciones orbitales de las posiciones de los objetos, cuyos errores no siempre son conocidos.\\

Conocer los errores en las posiciones es fundamental para el c\'alculo de la PoC, que es el par\'ametro actualmente aceptado para la clasificaci\'on del peligro o la toma de decisiones, como por ejemplo realizar una maniobra evasiva.\\ En este trabajo, para determinar las posiciones orbitales de los objetos se utilizaron los TLE p\'ublicos de la p\'agina Space-track, que no ofrece los errores asociados y el propagador SGP4 \citep{sgp4python} que tampoco da informaci\'on de los errores de propagaci\'on que introduce.\\

En ese sentido, a fin de mejorar la caracterizaci\'on de la situaci\'on de encuentro, se implement\'o el m\'etodo de Osweiler \citep{osweiler} para la estimaci\'on de las matrices de error de las posiciones de los objetos, y se utilizaron efem\'erides precisas para la elaboraci\'on de una tabla que contiene los valores de los errores propagados en funci\'on de los d\'ias de propagaci\'on.\\

Los resultados indican que la construcci\'on de las matrices de error se realiza correctamente, no obstante, los valores que arroja el m\'etodo son muy grandes y absorben los resultados encontrados para la matriz de propagaci\'on propuesta, impidiendo una mejora en la determinaci\'on de la posici\'on.  A su vez, el c\'alculo de la PoC es muy sensible a los valores de las matrices de covarianza, por lo que resulta insuficiente considerar las mejoras en la determinaci\'on de la propagaci\'on.\\

Para el c\'alculo de la PoC se estudiaron distintos m\'etodos, y se opt\'o por la expresi\'on simplificada que considera \'orbitas circulares que describe Lei-Chen \citep{leichen}. Puede observarse que esta expresi\'on resulta muy sensible a los errores de las posiciones de los objetos y al radio de colisi\'on seleccionado. As\'i, los resultados obtenidos en este trabajo, podr\'ian coincidir perfectamente con los datos publicados en el libro de  Klinkrad, \citep{Klinkrad} o con los resultados de Xi y Xiong \citep{xu2014method}), variando los valores del radio de colisi\'on, para aquellos casos en los que los errores en las posiciones no son muy grandes.

Con el objeto de incorporar todos estos c\'alculos a un procedimiento operativo frente a situaciones de riesgo de colisi\'on, se desarroll\'o un prototipo de software: ARxCODE, que recibe alertas de posibles colisiones en el formato estandarizado CDM o bien por ingreso manual del operador, y devuelve: el TCA, la m\'inima distancia entre objetos, la PoC y la proyecci\'on de las \'orbitas sobre la superficie de la Tierra; todo en una aplicaci\'on con una interfaz amigable; para dar soporte a los operadores en las situaciones de riesgo. 

ARxCODE fue pensado para operar montado sobre las estructuras ya existentes del departamento de Din\'amica Orbital y para ser utilizado por operadores o analistas con alto conocimiento sobre din\'amica orbital y desechos espaciales.\\

Entre sus funcionalidades principales, ARxCODE:\\

\begin{itemize}
 \item Extrae la informaci\'on que contienen los mensajes de alerta estandarizados CDM, y publica por pantalla: la {\it{m\'inima distancia}}, la {\it{probabilidad de colisi\'on}} y el {\it{tiempo de m\'aximo acercamiento}} (TCA). 
 \item Implementa el m\'etodo de Osweiler \citep{osweiler} para determinar las matrices de covarianza de los errores, en la posici\'on inicial.
 \item Implementa un m\'etodo desarrollado en esta tesis, para propagar las matrices de covarianza de errores hasta el TCA.
 \item Calcula la probabilidad de colisi\'on del encuentro.
 \item Grafica las trayectorias de los objetos durante el encuentro.
\end{itemize}


\section*{Trabajo a Futuro}

En este trabajo se logr\'o abarcar todo el recorrido desde un mensaje de alerta hasta la caracterizaci\'on de la situaci\'on del riesgo de una situaci\'on de encuentro; resultando en un prototipo software, con una interfaz amigable, pensado para ser utilizado por un operador con conocimientos de Din\'amica Orbital; no obstante, es necesario profundizar e incluso mejorar algunas de las instancias intermedias de la cadena de procesamiento, a fin de lograr resultados m\'as precisos y confiables; as\'i como una estructura que permita una implementaci\'on m\'as robusta con los datos que administra el departamento de Din\'amica Orbital\\

En primera instancia ser\'a necesario utilizar otros mecanimos o ajustar el m\'etodo propuesto para la determinaci\'on del error en la posici\'on inicial de los objetos. Entre las opciones que se han analizado, existe la posibilidad de incorporar una correcci\'on de ajuste a la posici\'on inicial que se utiliza en el c\'alculo del m\'etodo de Osweiler; esta correcci\'on se desprender\'ia de procesamientos previos que utilizan efem\'erides precisas. Ya se han realizado algunas pruebas, pero los resultados no fueron concluyentes, por lo que ser\'a necesario seguir evaluando de qu\'e modo impartir esa primera correcci\'on para esta idea.

Otra opción ..considerar un propagador preciso.


ARxCODE se pens\'o con una filosof\'ia de ampliaci\'on con el objeto de ir perfeccion\'andose con la propia experiencia en su utilizaci\'on. En este sentido, muchas de las pruebas e implementaciones no pudieron ser desarrolladas o validadas ya que por falta de tiempo o pol\'iticas de confidencialidad, no pudo tenerse acceso a cierta informaci\'on de interfaces o datos de situaciones reales.\\

, y validaciones con el acumulado de situaciones reales. No obstante, la imposibilidad  los datos puntuales de situaciones reales, dificult\'o las pr\'acticas de validaci\'on y muchas propuestas y desarrollos fueron descartados para ser abordados en un futuro.\\

de tener un mayor acceso a la din\'amica de las operaciones y ... la l\'ogica de las interfaces y la articulaci\'on de la recepeci\'on de los mensajes de alerta. 

\begin{itemize}
 \item Incorporar un ajuste que mejore la estimaci\'on de errores de las posiciones iniciales. 
 \item Incorporar al dise\~no una base de datos para los registros espec\'ificos de los CDM y los informes de alerta, o bien 
 crear la entidad asociada, dentro de la base existente en Din\'amica Orbital.
 \item Incorporar otros m\'etodos para el c\'alculo de la PoC.
\end{itemize}


\endinput