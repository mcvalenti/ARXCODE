 \chapter{Conclusiones}
\label{chap:conclusiones}

Como resultado de este trabajo se obtuvo el prototipo de software ARxCODE, para el an\'alisis de riesgo de posibles colisiones entre misiones operativas y desechos espaciales.\\
El mismo fue pensado para operar montado sobre las estructuras ya existentes del departamento de Din\'amica Orbital y para ser utilizado por operadores o analistas con alto conocimiento sobre din\'amica orbital y desechos espaciales.\\

ARxCODE es una aplicaci\'on que permite al operador visualizar la informaci\'on que contienen los mensajes internacionales estandarizados de alerta \ac{CDM} o bien, ingresar manualmente los datos de una situaci\'on de encuentro de riesgo, para su procesamiento.\\

Entre sus funcionalidades principales, el ARxCODE:\\

\begin{itemize}
 \item Extrae la informaci\'on que contienen los mensajes de alerta estandarizados CDM, y publica por pantalla: la {\it{m\'inima distancia}}, la {\it{probabilidad de colisi\'on}} y el {\it{tiempo de m\'aximo acercamiento}} (TCA). 
 \item Implementa el m\'etodo de Osweiler \cite{osweiler} para determinar las matrices de covarianza de los errores, en la posici\'on inicial.
 \item Implementa un m\'etodo desarrollado en esta tesis, para propagar las matrices de covarianza de errores hasta el TCA.
 \item Calcula la probabilidad de colisi\'on del encuentro.
 \item Grafica las trayectorias de los objetos durante el encuentro.
\end{itemize}

\section*{Trabajo a Futuro}

\begin{itemize}
 \item Incorporar un ajuste que mejore la estimaci\'on de errores de las posiciones iniciales. 
 \item Incorporar al dise\~no una base de datos para los registros espec\'ificos de los CDM y los informes de alerta, o bien 
 crear la entidad asociada, dentro de la base existente en Din\'amica Orbital.
 \item Incorporar otros m\'etodos para el c\'alculo de la PoC.
\end{itemize}


\endinput