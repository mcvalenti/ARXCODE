 \chapter{Conclusiones}
\label{chap:conclusiones}

En esta tesis se analiz\'o la problem\'atica de los desechos espaciales con foco central en los riesgos de colisi\'on con misiones operativas. El estudio de un acercamiento de riesgo o encuentro, se basa en los resultados de las propagaciones orbitales de las posiciones de los objetos, cuyos errores no siempre son conocidos, por lo que no es suficiente determinar la m\'inima distancia entre los objetos, sino que adem\'as es necesario hacer estudios sobre la probabilidad de colisi\'on, PoC.\\

Conocer los errores en las posiciones de los objetos es fundamental para el c\'alculo de la PoC, que es el par\'ametro actualmente aceptado para la clasificaci\'on del peligro o la toma de decisiones, como por ejemplo realizar una maniobra evasiva.\\ 

En este trabajo, para determinar las posiciones orbitales de los objetos se utilizaron los TLE p\'ublicos de la p\'agina Space-track, que no ofrece los errores asociados y el propagador SGP4 \citep{sgp4python} que tampoco da informaci\'on de los errores de propagaci\'on que introduce.\\

En ese sentido, a fin de obtener la PoC y mejorar la caracterizaci\'on de la situaci\'on de encuentro, se implement\'o el m\'etodo de Osweiler \citep{osweiler} para la estimaci\'on de las matrices de error de las posiciones de los objetos, y se utilizaron efem\'erides precisas para la elaboraci\'on de una tabla que contiene los valores de los errores en las propagaciones en funci\'on de los d\'ias de propagaci\'on, que incluye hasta 6 d\'ias hacia el futuro.\\

Los resultados indican que la construcci\'on de las matrices de error se realiza correctamente, no obstante, los valores que arroja el m\'etodo son muy grandes y absorben los resultados encontrados para la matriz de propagaci\'on propuesta. Se realizaron pruebas que calculan la PoC considerando estas matrices generadas y tambi\'en otras de valores medios generales que resultan de estudios emp\'iricos sobre \'orbitas LEO de caracter\'isticas similares a las estudiadas en esta tesis, como las matrices propuestas por Flohrer et al., \citep{flohrer2008assessment} y por el servicio web SOCRATES \citep{Kelso}. Los valores que se obtienen de la PoC con matrices de estos \'ultimos trabajos son mejores, por lo que es necesario ajustar los resultados del m\'etodo de Osweiler o directamente reemplazarlos.\\

Las m\'inimas distancias calculadas, no siempre est\'an en concordancia con las que se indican en los distintos casos de prueba, ya sea CDM, correos electr\'onicos o los diez casos utilizados para validar la PoC. Se atribuye esta diferencia a distintas condiciones iniciales (TLE de distintas \'epocas por ejemplo) y a la utilizaci\'on de distintos modelos de propagaci\'on en la predicci\'on de los acercamientos, pero desconociendo el modo en que estos son generados para los casos de prueba, no es posible hacer una afirmaci\'on respecto de este punto.\\ 

Para el c\'alculo de la PoC se implementaron tres m\'etodos de distinta complejidad. En primer lugar se utiliza una expresi\'on muy simplificada que no contempla los errores de las posiciones, m\'etodo del l\'imite  \citep{alfano2008method}. Luego se utiliza una aproximaci\'on del m\'etodo de Chan \citep{chan2003improved} para \'orbitas circulares propuesta por Lei-Chen \citep{leichen} y finalmente se calcula la integral de la PoC en dos dimensiones utilizando el m\'etodo de Akella \citep{akellaAlfriend}. El m\'etodo del l\'imite ofrece resultados que siempre sobrestiman el valor de la PoC en uno o dos \'ordenes de magnitud tendiendo a generar falsas alarmas permanentemente. Por el contrario el m\'etodo de Lei-Chen siempre calcula valores menores para la PoC, con el riesgo de desestimar situaciones que pueden ser de riesgo, no obstante mejora su precisi\'on cuando se reduce el valor de los errores considerados. El m\'etodo de Akella \& Alfriend es el que mejor parece responder cuando se consideran errores peque\~nos y es el que mayor sensibilidad muestra a la relaci\'on entre los errores y la m\'inima distancia entre los objetos.\\

Con el objeto de incorporar todos estos c\'alculos a un procedimiento operativo frente a situaciones de riesgo de colisi\'on, se desarroll\'o un prototipo de software: ARxCODE, que recibe alertas de posibles colisiones en el formato estandarizado CDM o bien por ingreso manual del operador, y devuelve: el TCA, la m\'inima distancia entre objetos, la PoC y la proyecci\'on de las \'orbitas sobre la superficie de la Tierra; todo en una aplicaci\'on con una interfaz amigable; para dar soporte a los operadores en las situaciones de riesgo.\\

ARxCODE fue pensado para operar montado sobre las estructuras ya existentes del departamento de Din\'amica Orbital y para ser utilizado por operadores o analistas con alto conocimiento sobre din\'amica orbital y desechos espaciales.\\

Entre sus funcionalidades principales, ARxCODE:

\begin{itemize}
 \item Extrae la informaci\'on que contienen los mensajes de alerta estandarizados CDM, y publica por pantalla: la {\it{m\'inima distancia}}, la {\it{probabilidad de colisi\'on}} y el {\it{tiempo de m\'aximo acercamiento}} (TCA). 
 \item Implementa el m\'etodo de Osweiler \citep{osweiler} para determinar las matrices de covarianza de los errores, en la posici\'on inicial.
 \item Implementa un m\'etodo desarrollado en esta tesis, para propagar las matrices de covarianza de errores hasta el TCA.
 \item Calcula la m\'inima distancia y la probabilidad de colisi\'on del encuentro.
 \item Grafica las trayectorias de los objetos durante el encuentro.
\end{itemize}


\section*{Trabajo a Futuro}

En este trabajo se logr\'o abarcar todo el recorrido desde un mensaje de alerta hasta la caracterizaci\'on de la situaci\'on del riesgo de una situaci\'on de encuentro. El resultado es un prototipo software, con una interfaz amigable, pensado para ser utilizado por un operador con conocimientos de Din\'amica Orbital. Sin embargo, es necesario profundizar e incluso mejorar algunas de las instancias intermedias del flujo de procesamiento, a fin de lograr resultados m\'as precisos y confiables; as\'i como una estructura que permita una implementaci\'on m\'as robusta con los datos que administra el departamento de Din\'amica Orbital, que reemplace, por ejemplo, la utilizaci\'on de directorios locales para el almacenamiento de las efem\'erides precisas, los TLE y los mensajes de alerta, por una base de datos.\\

En cuanto a los resultados, en primera instancia ser\'a necesario utilizar otros mecanimos o ajustar el m\'etodo propuesto para la determinaci\'on del error en la posici\'on inicial de los objetos. Entre las opciones que se han analizado, existe la posibilidad de incorporar valores medios como los que se proponen en la bibliograf\'ia o implementar una correcci\'on de ajuste a la posici\'on inicial que se utiliza en el c\'alculo del m\'etodo de Osweiler \citep{osweiler}; esta correcci\'on se desprender\'ia de procesamientos previos que utilizan efem\'erides precisas de la misi\'on operativa. En esta direcci\'on, ya se han realizado algunas pruebas, pero los resultados no fueron concluyentes, por lo que ser\'a necesario seguir evaluando de qu\'e modo impartir esa primera correcci\'on para lograr mejoras en la determinaci\'on de la posici\'on inicial.\\

Otra opci\'on posible, consiste en utilizar un propagador preciso y t\'ecnicas de ajuste que mejoren los resultados de la comparaci\'on de pares para la generaci\'on de las matrices para el error inicial. Esta opci\'on, implica el desarrollo de un propagador preciso o la utilizaci\'on de propagadores ya existentes, cuyos procesamientos deben ser incorporados al ARxCODE; esto no fue posible realizarlo en el transcurso de este trabajo por cuestiones de tiempo.\\

El hecho de no contar con datos de situaciones reales para la misi\'on operativa cuyas efem\'erides precisas fueron utilizadas, fue una gran limitante para  el estudio de las distintas t\'ecnicas y validaciones. No obstante, ARxCODE se dise\~n\'o y desarroll\'o con facilidades para ser modificado y perfeccionado a partir del estudio de los resultados que el mismo ofrezca en situaciones reales durante un largo periodo de prueba. Es decir que, a medida que puedan incorporarse nuevos escenarios de validaci\'on, ya sea de la propia CONAE o de publicaciones que ofrezcan datos precisos, se podr\'an probar nuevas t\'ecnicas de correcci\'on en las posiciones y m\'etodos para el c\'alculo de la PoC.\\

El c\'alculo de la PoC puede perfeccionarse incluyendo m\'as t\'erminos de la expresi\'on de Lei-Chen \citep{leichen} o ampliarse a situaciones m\'as generales que incluyan objetos con \'orbitas no circulares. El estudio de los valores que ofrece la implementaci\'on del m\'etodo de Akella \& Alfriend puede mejorarse agregando el an\'alisis de la relaci\'on entre los errores y la m\'inima distancia estipualda. ARxCODE podr\'ia a su vez incorporar otros de los algoritmos que propone el CCSDS para el c\'alculo de la PoC (Alfano \citep{alfano2008method}, Foster \citep{foster}, etc) y as\'i ir perfeccionando sus estimaciones. Implementaciones que no se realizaron en el contexto de este trabajo por cuestiones de tiempo.\\


\endinput