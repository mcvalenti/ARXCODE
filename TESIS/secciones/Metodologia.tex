\chapter{Metodología de Desarrollo}
\label{chap:metodologia}

 %De acuerdo a \cite{Barrett2009}, el modelo WCM...\\
Para el desarrollo de este trabajo se opt\'o por una metodolog\'ia del tipo incremental.
En cada iteraci\'on, se realizan cambios en el dise\~no y se agregan nuevas funcionalidades y capacidades al sistema.
\begin{itemize}
\item En el desarrollo de este modelo se da la retroalimentaci\'on muy temprano a los usuarios.
\item Permite separar la complejidad del proyecto, gracias a su desarrollo por parte de cada iteraci\'on o bloque.
\item El producto es consistente y puntual en el desarrollo.
\item Los productos desarrollados con este modelo tienen una menor probabilidad de fallar.
\item Se obtiene un aprendizaje en cada iteraci\'on que es aplicado en el desarrollo del producto y aumenta las experiencias para pr\'oximos proyectos.
\end{itemize}


\section{Inicializaci\'on}
En esta primera etapa evaluamos  el concepto del ARxCODE en el contexto de la Unidad de Desarrollo de Desechos Espaciales de CONAE. Fundamentalmente la vinculaci\'on con el departamento de Din\'amica Orbital y los procedimientos actuales que se realizan en realaci\'on a los riesgos de colisi\'on con desechos. Hicimos un estudio de las estructuras org\'anicas existentes y los sistemas asociados. Los distintos tipos de productos y usuarios, las interfaces que existen y el acceso a los datos reales con los que podr\'iamos contar.\\
Analizamos como trabajan otras agencias espaciales la tem\'atica y concluimos qu\'e es lo que podr\'ia ofrecerse y bajo qu\'e premisas.
De las consideraciones m\'as importantes que se desprendieron de esta etapa cabe destacar que se decidi\'o un prototipo para funcionar montado sobre el software principal de Din\'amica Orbital, como un anexo que no interfiere de ninguna manera con los procesos actuales.\\
Por otro lado, debido a la complejidad del problema y sus consecuencias, ser\'a un software diseñado para ser utilizado por un analista experto, con conocimientos de Din\'amica Orbital. En el mismo sentido, sus resultados no ser\'an considerados en la toma de decisiones hasta tanto sus resultados no hayan sido validados durante un periodo suficiente, que permita verificar y mejorar su funcionamiento, valid\'andolo con un acumulado de situaciones reales.\\
Para este planteo de definiciones, contamos con el asesoramiento y el intercambio de informaci\'on con personas de Din\'amica Orbital y otros departamentos de CONAE. Realizamos algunas reuniones e intecambios de mails, aunque por ser una tem\'atica que se aborda bajo reg\'imenes especiales de acuerdos de confidencialidad, no fue posible contar con la totalidad de la informaci\'on.

\section{Iteraci\'on}
Ya conocido el planteo del problema, las distintas maneras de abordarlo y las restricciones que resultan, elaboramos un diseño perliminar del producto con sus requerimientos (ver ref secci\'on de requerimientos) y sus funcionalidades, que dadas las caracter\'isticas del problema resultan bastante deterministas.\\

Para el desarrollo definimos distintas estructuras o paquetes que agrupan tipos de datos o entradas con el mismo formato y procedimientos particulares.\\
\begin{itemize}
 \item Paquetes de Administraci\'on de las Entradas (TLE, CODS ephem, CDM), propios del tipo de dato y el tratamiento asociado.
 \item Paquetes de procesamiento de las Entradas.
 \item Paquetes Generales de utilizaci\'on m\'ultiple: Sistema de Referencia, Visual, Estadistica.
 \item Paquetes de procesamiento de niveles m\'as avanzados: Encuentro. 
 \item Paquete que genera la interfaz.
\end{itemize}

Esta metodolog\'ia nos permiti\'o importar funciones que resuelven cuestiones espec\'ificas desde cualquier  m\'odulo y a su vez modificar las funciones cuando fuera necesario, sin que ello implicara grandes cambios en la estructura global. En especial, cuando empezamos con el diseño y el desarrollo de la interfaz, fuimos modificando mucho las opciones, en tanto utilizamos la interfaz para seguimiento de pasos intermedios que a medida que iban siendo validados ibamos quit\'andolas de las opciones del usuario.\\

\section{Control}
Control sobre los datos. (En particular para SAC-D - Documentaci\'on interna (ICD y presentaci\'on en Meeting))\\
Control sobre los algoritmos implementados en base a resultados publicados\\
Unitest del flujo de datos.\\
Reuniones peri\'odicas de validaci\'on de an\'alisis de resultados y pr\'oximos pasos.\\



\section{Entorno de Desarrollo}
En el IDE Eclipse\\
Lenguaje Python y PyQT.\\
Hablar de Python y de QT.\\
librerias instaladas y reutilizaci\'on. (SGP4)\\
Control de Versiones Git.\\
Unitest y escenarios de validaci\'on.


