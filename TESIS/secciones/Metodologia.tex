\chapter{Metodología}
\label{chap:metodologia}

 De acuerdo a \cite{Barrett2009}, el modelo WCM...\\

 
 
 %  Ejemplo para citar otros capítulos o secciones
  Como dice el capítulo \ref{chap:introduccion}...

\subsection{Tratamiento sobre Datos de Misi\'on}

\subsubsection{Ma. de Covarianza Simplificada}
Se construye a partir de las diferencias entre las efem\'erides predichas por CODS y los vectores de estado que resultan de los TLEs.\\


\subsubsection{Ma. de Covarianza Ajustada}
Se construye a partir de las diferencias recalculadas a partir de los vectores de estado que resultan de los TLEs, corregidos por el ajuste.\\


\subsubsection{Procedimiento de Ajuste}
Consideramos la Misi\'on SAC-D.

Contamos con acceso a los datos de las efem\'erides predichas calculadas por CODS, pero no están publicados en estos archivos los errores asociados al c\'alculo.\\
Nos proveemos de los TLE de la Misi\'on para la misma \'epoca.\\
Evaluamos las diferencias entre los distintos vectores de estado (el predico por CODS y el que resulta de TLE).\\

{\bf{M\'etodo para la estimaci\'on del error en los datos extrapolados.}}\\
Se definen dos intervaloes: uno para el ajuste, otro para la extrapolaci\'on.\\
Se ajustan las diferencias y se obtiene una funci\'on de ajuste.\\
Se eval\'ua la funci\'on de ajuste en las fechas correspondientes a los TLEs del intervalo de extrapolaci\'on.\\
{\textcolor{red}{Se interpolan los valores de CODS para las fechas de los TLE del intervalo de extrapolaci\'on.}}\\
Se ajustan los valores de los vectores de estado de los TLE del intervalo de extrapolaci\'on, utilizando los valores que ofrece la funci\'on de ajuste. (En este punto se pretende que las posiciones y velocidades de los TLEs se acerquen a las posiciones y velocidades que ofrece CODS.)\\

