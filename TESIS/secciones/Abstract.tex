\chapter*{Abstract}
\label{chap:abstract}



In this thesis we present ARxCODE software prototype, an operational system to monitor close conjunctions between operational satellites and space debris. ARxCODE process risk encounter alerts messages under international CCSDS (Consultative Committee for Space Data System) standard CDM (Conjunction Data Messagge) produced by JSpOC (Joint Space Operations Center) and returns the  probability of collision as well as a complete encounter scenario caracterization and visualization.
Risk conjunctions analysis, in particular those with space debris, involves significant position uncertainties especially for agencies without space surveillance networks. Nowadays, besides minimum distance, probability of collision is required.
Covariance matrices are estimated with Osweiler method plus errors determination adjustment using precise orbit reference. Historic SAC-D precise orbit products are used in order to evaluate position uncertainty trends and improve errors determination in the TLE orbit propagations using Simplified General Perturbations analitical model (SGP4). Finally three different algorithms are implemented for the probability of collision computation. 

This tool offers mission control center operators the possibility to have a clearer overview of the conjunction situation, especially in dialog with international agencies during risk situation intervals.


