\chapter{Marco Teórico}
\label{chap:marcoteorico}

% Estado del arte general de la temática de estudio. 
\section{Introducci\'on}

La estimaci\'on de errores en la determinaci\'on orbital en la base de el procedimiento.(redactar mejor)\\
En un contexto ideal, de una Tierra esf\'erica, homog\'enea y sin rozamiento, los modelos f\'isicos que describen el movimiento de los objetos que orbitan la Tierra, nos permitir\'ian predecir con much\'isima precisi\'on las sitaciones de colisi\'on. Es decir, si conoci\'eramos muy bien la posici\'on de las naves y las condiciones del entorno, y pudi\'eramos modelarlo, podr\'iamos hacer predicciones futuras de las posciones y as\'i determinar si los objetos van a colisionar o no sin ambig\"uedad. Pero esa no es la realidad.\\
La Tierra no es perfectamente esf\'erica y mucho menos homog\'enea, y si bien es la principal influencia en las \'orbitas de las naves, existen otros cuerpos como la Luna, el Sol y los dem\'as planetas que tambi\'en ejercen fuerzas gravitatorias en mayor o menor medida.\\
Existen tambi\'en otras fuerzas, que resultan del entorno, como por ejemplo el frenado de la atm\'osfera para los objetos de \'orbitas LEO y la presi\'on de radiaci\'on solar que afecta principalmente a los sat\'elites con grandes paneles.\\

Todos estos condimentos, nos muestran que la determinaci\'on de las posiciones de los objetos que orbitan la Tierra, no es una tarea sencilla. Dependiendo de los distintos m\'etodos y modelos de determinaci\'on orbital con los que se cuentan, podremos saber con mejor o peor precisi\'on la posici\'on de los objetos.El estudio de riesgo de colisi\'on se basa en la capacidad de determinar y predecir las \'orbitas de los objetos involucrados en un acercamiento con la mayor precisi\'on posible.\\.
Esto implica un aspecto crucial en el estudio de las posibles colisiones, ya que cuando se pretenden hacer predicciones orbitales, debemos saber con qu\'e margen de error hemos calculado la sitauci\'on de acercamiento. Y es esta misma raz\'on la que explica por qu\'e en el c\'alculo de colisiones se habla de probabilidad de colisi\'on (PoC)\\

Mejorar estas estimaciones es fundamental, ya que en base a ellas se realizan o no maniobras, que, por un lado interfieren en los experimentos cient\'ificos que se llevan a cabo en las distintas plataformas, las maniobras son siempre operaciones delicadas, y en las misiones con astronautas a bordo (como por ejemplo la Estaci\'on Espacial Internacional (ISS)), una decisi\'on err\'onea pone en riesgo a la tripulaci\'on.\\

Por otra parte, distintos estudios realizados (... Foster ¿?), muestran que existen en varios m\'etodos actuales de c\'alculo de PoC, sobreestimaciones de los acercamientos y los riesgos, generando falsas alarmas que de no ser revisadas con minuciosidad conducen a decisiones de maniobras innecesarias. 

El epoca de la colisi\'on y la previsi\'on en los c\'alculos.(redactar)\\
La determinaci\'on orbital es m\'as f\'acil a posteriori.\\

\section{La determinaci\'on orbital de los objetos involucrados}

En nuestro planteo de riesgos por colisi\'on entre misiones operativas y desechos espaciales, existir\'an dos planteos distintos del problema de la determinaci\'on de las posiciones, ya que cada uno de los objetos involucrados ofrece metodolog\'ias y modelos diferentes.

\subsection{La posici\'on de la misi\'on operativa}
Con la misi\'on operativa tenemos contacto y comandado. En gral tiene un GPS o alg\'un otro instrumento de determinaci\'on de la posici\'on, con un cierto error asociado, que es conocido.({\bf{cu\'al?}})\\
A su vez CONAE tiene su modelo orbital de propagaci\'on y generaci\'on de predicci\'on (PREPHEM), y su propio modelo de ajuste (ORBEPHEM).\\

\subsection{La posici\'on del desecho espacial}
El desecho espacial no tiene capacidades operativas, de manera que la \'unica manera de determinar su posici\'on es con redes de rastreo desde Tierra.\\
Caracter\'istica de las redes de rastreo. error asociado, que es conocido.({\bf{cu\'al?}})\\
S\'olo usa, rusia y la uni\'on europea cuentan con redes de rastreo. En particular Norad es la que disponibiliza en forma p\'ublica los datos en formato TLE.\\
Dado este contexto, las distintas agencias, que no cuentan con gran n\'umero de misiones operativas, suelen contratar/acordar este servicio. Y a\'un, con herramientas propias desarrolladas, lleva tiempo la validaci\'on de las mismas hasta lograr cierta autonom\'ia. No obstante, la tendencia es hacia un primer paso, que permita una caracterizaci\'on m\'as precisa de la situaci\'on de encuentro, sumando a la informaci\'on p\'ublica, m\'etodos de ajuste, datos propios de la misi\'on en riesgo y estimadores del riesgo como por ejemplo la \ac{PoC}.\\
Es en esa direcci\'on que planteamos este trabajo, a fin de ofrecer un prototipo que pueda ser implementado en paralelo a lo que ya se utiliza en CONAE, para ser probado y mejorado con la experiencia que sumen los encuentros que involucren a las misiones nacionales.\\
Describir:\\
\begin{itemize}
\item NORAD.
\item TLE.
\item SGP4.
\end{itemize}


\section{El riesgo de colisi\'on}
Un an\'alisis completo del Riesgo de Colisi\'on, abarca: (esquematizar)

\begin{itemize}
\item Identificar las situaciones de encuentro.
\item Analizar la situaci\'on del encuentro.
\item Ejecutar maniobras de mitigaci\'on del riesgo si fuera necesario.
\item Iterar el proceso con minuciocidad para no ofrecer soluciones moment\'aneas que generen nuevos riesgos de colisi\'on.
\end{itemize}

\subsection*{Identificaci\'on de las situaciones de encuentros:}
A partir de los datos generados por las redes de rastreo, se propagan las trayectorias orbitales y, bajo ciertos criterios definidos previamente, se detectan los acercamientos no deseados. En esta idea subyace la definici\'on de {\it{Encuentro}}.\\
Son pocos los organismos y agencias capaces de realizar este procedimiento, incorporando a sus predicciones todos los objetos catalogados y/o rastreados.\\
En un formato m\'as simplificado, el inter\'es se enfoca en una misi\'on en particular y se desarrollan filtros, para procesar encuentros analizando una menor cantidad de objetos.\\
\subsubsection*{CDM}
(JAC SW - Laporte)
{\bf{CSM:}} They are made available on Emergency Criteria, wich are Time of Closest Approach whithin 72 hs combined with a miss distance criteria:\\
LEO:\\
overall miss distance < 1km\\
radial miss distance < 200m\\
GEO/MEO:\\
Overall miss distance < 10 km.\\

CSM are advisory and informational messages only and are not directly actionable. They don´t provide a direct recommendation to perform an avoidance action and of course they cannot take neither the operational constraints of the asset nor the maneuvers the asset plansor just performed. (sigue ver apuntes Laporte carpeta...)\\

!!! JSpOC NO CUENTA CON INFORMACI\'ON DE MANIOBRAS PLANIFICADAS, puede haber falsas alarmas.\\

\subsection*{An\'alisis de la situaci\'on del encuentro: }
El mismo consiste en estudiar el panorama con mayor profundidad y detalle, sumando informaci\'on m\'as confiable en la determinaci\'on orbital, y calculando par\'armetros estad\'isticos, como la Probabilidad de Colisi\'on (PoC).\\
A medida que se aproxima la fecha en la que se predice el encuentro, se tiene mejor conocimiento de la \'orbita de los objetos involucrados, pero menor tiempo de reacci\'on en la toma de decisiones. Es decir, en el an\'alisis del encuentro se busca un balance entre los tiempos que conlleve el estudio para alcanzar la confiabilidad necesaria, y el margen que se requiere para, por ejemplo, planificar una maniobra.\\
En este item en particular se enfoca este trabajo.\\

Metodolog\'ia Akella \& Alfriend.\\

\subsection*{realizaci\'on de una maniobra}
Si la situaci\'on lo ameritara, la \'unica manera de evitar una colisi\'on es la {\bf{realizaci\'on de una maniobra}}, conocidas como Maniobras de Mitigaci\'on de Riesgo (RMM). No obstante, modificar la trayectoria de un objeto, siempre presupone una evaluaci\'on a priori de que no vaya a producirse una colisi\'on. De manera, que en este punto, se vuelve al item inicial y se repite el proceso iterativamente.\\ 







