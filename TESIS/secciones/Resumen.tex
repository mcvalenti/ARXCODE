\chapter*{Resumen}
\label{chap:resumen}
En este trabajo se presenta el prototipo de software ARxCODE: {\it{An\'alisis de Riesgo por Colisi\'on con Desechos Espaciales}}. Un sistema dise\~nado para el estudio de acercamientos con riesgo de colisi\'on, entre misiones satelitales operativas y desechos espaciales. 
ARxCODE tiene la capacidad de extraer la informaci\'on que proviene de los mensajes de alerta de colisiones estandarizados, \ac{CDM}, definidos por el \ac{CCSDS} y de procesar datos ingresados manualmente.
Devuelve al operador par\'ametros para el an\'alisis de riesgo como: la m\'inima distancia y la probabilidad de colisi\'on  o \ac{PoC} en una interfaz gr\'afica que facilita una clara caracterizaci\'on y visualizaci\'on de la situaci\'on.

Los estudios de acercamientos de riesgo que involucran desechos espaciales acarrean grandes incertezas respecto a la posici\'on del desecho, en especial para aquellos organismos que no cuentan con instrumentos propios de rastreo.
En la actualidad, adem\'as de la distancia m\'inima de acercamiento, debe considerarse la probabilidad de colisi\'on. El c\'alculo de la probabilidad de colisi\'on requiere tener conocimiento de los errores de las posiciones y esto no siempre es conocido, en particular para los desechos, cuyos \ac{TLE} son p\'ublicos, pero no sus errores asociados. 

Para la construcci\'on de la matriz de covarianza de la posici\'on del desecho correspondiente al \'ultimo TLE disponible m\'as cercano al momento del m\'aximo acercamiento \ac{TCA}, se implementa el m\'etodo desarrollado por Osweiler \citep{osweiler}.
Para la estimaci\'on de los errores que introducen las propagaciones de los TLE, con el modelo de propagaci\'on anal\'itico \ac{SGP4} se desarroll\'o una metodolog\'ia que incorpora el análisis hist\'orico de los productos orbitales precisos de la misi\'on SAC-D.
Finalmente para el c\'alculo de la probabilidad de colisi\'on, se implementaron tres m\'etodos con complejidades diferentes: m\'etodo del l\'imite \citep{alfano2008method}, m\'etodo de Lei-Chen, \citep{leichen} y m\'etodo de Akella \& Alfriend \citep{akellaAlfriend}.

ARxCODE es una herramienta que ofrece a los operadores de los centros de control de misi\'on, una visi\'on m\'as clara de las situaciones de encuentro, para los momentos de di\'alogo e intercambio de informaci\'on con los organismos internacionales de alerta.

%\textbf{Palabras clave:}:space debris --- collision avoidance --- probability of collision
\endinput
%%%% Texto (no menos de 200 palabras)


La capacidad de predecir y evitar el riesgo de colisión de un sat\'elite operativo con un desecho espacial, se ve limitada por los errores de precisi\'on con los que se conocen las \'orbitas de los desechos espaciales.\\

En la actualidad s\'olo EE.UU y Rusia, y la comunidad europea dando sus primeros pasos, cuentan con redes e instrumentos de rastreo que les permiten confeccionar bases de datos con la informaci\'on orbital de los objetos que rodean la Tierra.\\

En este contexto, las maniobras que involucren a las misiones de la Comisi\'on Nacional de Actividades Espaciales (CONAE) con riesgo de colisi\'on, se planifican a partir de la informaci\'on que le proveen servicios externos a trav\'es de convenios con alto grado de confidencialidad e informaci\'on restringida.\\ 

Organismos como el \ac{JSpOC} de EE.UU, tienen la capacidad de predecir encuentros y los notifica con una antelaci\'on de 72 horas, a trav\'es de sus mensajes de alerta: \ac{CDM} \cite{CDM}.\\
Estos mensajes proveen informaci\'on de los objetos involucrados en el encuentro, las posiciones y velocidades de ambos, el instante predicho para el encuentro, \ac{TCA} y las matrices de error para el TCA.\\

\textcolor{red}{A partir de los datos que ofrece el \ac{CDM}, nos proponemos desarrollar una t\'ecnica que mejore la estimaci\'on de la posici\'on del desecho espacial, e incorpore los datos orbitales que el departamento de Din\'amica Orbital ofrece en el c\'alculo de la \ac{PoC}, para contar con una mejor caracterizaci\'on de la situaci\'on al momento del intercambio de informaci\'on con los organismos y servicios externos.}\\
