\chapter*{Resumen}
\label{chap:resumen}

\endinput
%%%% Texto (no menos de 200 palabras)

%\textbf{Palabras clave:}

La capacidad de predecir y evitar el riesgo de colisión de un sat\'elite operativo con un desecho espacial, se ve limitada por los errores de precisi\'on con los que se conocen las \'orbitas de los desechos espaciales.\\

En la actualidad s\'olo EE.UU y Rusia, y la comunidad europea dando sus primeros pasos, cuentan con redes e instrumentos de rastreo que les permiten confeccionar bases de datos con la informaci\'on orbital de los objetos que rodean la Tierra.\\

En este contexto, las maniobras que involucren a las misiones de la Comisi\'on Nacional de Actividades Espaciales (CONAE) con riesgo de colisi\'on, se planifican a partir de la informaci\'on que le proveen servicios externos a trav\'es de convenios con alto grado de confidencialidad e informaci\'on restringida.\\ 

Organismos como el \ac{JSpOC} de EE.UU, tienen la capacidad de predecir encuentros y los notifica con una antelaci\'on de 72 horas, a trav\'es de sus mensajes de alerta: \ac{CDM} \cite{CDM}.\\
Estos mensajes proveen informaci\'on de los objetos involucrados en el encuentro, las posiciones y velocidades de ambos, el instante predicho para el encuentro, \ac{TCA} y las matrices de error para el TCA.\\

\textcolor{red}{A partir de los datos que ofrece el \ac{CDM}, nos proponemos desarrollar una t\'ecnica que mejore la estimaci\'on de la posici\'on del desecho espacial, e incorpore los datos orbitales que el departamento de Din\'amica Orbital ofrece en el c\'alculo de la \ac{PoC}, para contar con una mejor caracterizaci\'on de la situaci\'on al momento del intercambio de informaci\'on con los organismos y servicios externos.}\\
